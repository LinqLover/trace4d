% !TeX root = ../paper.tex
\section{Visualization Approach}
\label{sec:visualization_approach}

To support the comprehension of object-oriented programs, we propose \emph{animated 2.5D object maps} as a novel visualization approach for program traces.
In the following, we describe the prerequisites and the design of our approach.

\subsection{Data Model}
\label{sec:visualization_approach/data_model}

\begin{figure}
	\includegraphics[width=\linewidth]{sections/03_visualization_approach/data_model.tikz}
	\caption{UML class diagram showing the data model of an object-oriented program trace for the visualization.}
	\Description{TODO}
	\label{fig:visualization_approach/data_model}
\end{figure}

%For the data source of the visualization, we assume a simple program trace model for object-oriented programs~(\cref{fig:visualization_approach/data_model}).
The data source of our visualization is the program trace of an object-oriented program.
%An object is characterized by its \emph{identity} which distinguishes it from all other objects in the system, its \emph{state} which is represented by its fields such as array elements and instance variables, and its \emph{behavior} which is described by methods that implement the reception of messages~\cite{thiede2023time}.
In this programming paradigm, all behavior is described as \emph{messages} sent from one object to another.
Each object is characterized by its \emph{identity} which distinguishes it from all other objects in the system, its \emph{state} which is represented by its fields such as array elements and instance variables, and its \emph{behavior} which is implemented by methods that are invoked to receive messages~\cite{thiede2023time}.

We assume a minimal data model of the program trace~(\cref{fig:visualization_approach/data_model}):
the \emph{call tree} is represented as a composite structure of \emph{stack frames} each of which specifies a time interval, an invoked method, and a receiver object.
Each \emph{object} is assigned a label, a list of named fields, and a class.
%The value of each \emph{field} can be a reference to another object or a flat string representation.
Each \emph{class} is described by a name and an organizational path in the file or package structure of the software system.
We neglect runtime changes to the state, label, or class membership of objects as well as metaprogramming peculiarities such as the implementation of classes or methods as objects.
% TODO: we also neglect coroutine, multiple processes, etc. but is this the right place to mention all of that?

\subsection{Visual Mapping}
\label{sec:visualization_approach/mapping}

We describe the design of our visualization and the mapping of parts from the program trace to elements and visual variables of our visualization~(\cref{fig:teaser}).
At the highest level, an animated 2.5 object map is an interactive information landscape that displays objects and their interactions from the program trace.
Users can replay the program trace and watch the \emph{activation} of objects, i.e., the execution of any of their methods, and their \emph{interaction}, i.e., the exchange of messages between two objects.
They can navigate freely through the visual scene using their keyboard and pointing devices and view the map from all sides.

\paragraph{Objects}
\label{sec:visualization_approach/mapping/objects}

\begin{figure}
	\includegraphics[width=\linewidth]{sections/03_visualization_approach/mapping/objects}
	\caption{Visual mapping of objects, fields, and references to block entities, tiles, and arrows in the object map.}
	\Description{TODO}
	\label{fig:visualization_approach/mapping/objects}
\end{figure}

Each object is represented as a square cuboid \emph{block} entity that displays the label and fields of the object~(\cref{fig:visualization_approach/mapping/objects}).
To maximize legibility from any perspective, the label is repeated on all four sides and in four orientations on the top of the block.
Fields are displayed as \emph{tiles} that are arranged in a row-wise uniform-sized grid layout and repeated on each side of the block for better legibility.
References between objects are rendered as \emph{directed arrows} from the closest tile of the referencing field to the closest label of the referenced object's entity.
To indicate the direction of arrows, we place between one and ten evenly distributed \emph{chevrons} on the arrow line; each chevron is displayed as a cone whose direction can be recognized from any perspective.

\paragraph{Object graph}
\label{sec:visualization_approach/mapping/object_graph}

All object blocks are placed on a plane in the 2.5D object map.
For their arrangement, we define a force-directed graph layout~\cite{fruchterman1991graph}.
Between each pair of object blocks $a$ and $b$, we apply several \emph{weighted attractive forces} based on the class membership ($F_\text{class}$), the organizational proximity of classes ($F_\text{org}$), and the references ($F_\text{ref}$) and communication ($F_\text{comm}$) between objects.
In the following definitions, the respective $w$ denote the weight of each force, and $\text{org}(o)$ denotes the organizational path of an object $o$'s class (e.g., a file path):

%\begin{equation}
%	\begin{split}
%		F_{\text{class}}(a, b) &= w_{\text{class}}\left(\begin{cases}1, & \text{if $\text{class}(a) = \text{class}(b)$;} \\ 0, & \text{otherwise}.\end{cases}\right) \,, \\
%		F_{\text{org}}(a, b) &= w_{\text{org}}\bigl(\text{LCP}\footnotemark\bigl(\text{org}\footnotemark(a), \text{org}(b)\bigr)\bigr) \,, \\
%		F_{\text{ref}}(a, b) &= w_{\text{ref}}\left(\left|\bigl\{ (k, v) \in \text{fields}(a) ~\middle\vert~ v = b \bigr\}\right|\right) \,, \\
%		F_{\text{comm}}(a, b) &= w_{\text{comm}}\big(\big|\big\{\text{frame }f ~\big\vert \\
%			& \hphantom{= w} \text{$f$.receiver} = a \wedge \text{$f$.parent.receiver} = b\big\}\big|\big) \, .
%	\end{split}
%\end{equation}
%\addtocounter{footnote}{-1}
%\footnotetext{$\text{LCP}(u, v)$: Largest common prefix of two sequences $u$ and $v$.}
%\footnotetext{$\text{org}(o)$: Organizational path to an object $o$'s class (e.g., a file path).}

\begin{algorithm}
	$F_{\text{class}}(a, b) = \begin{cases}w_{\text{class}}, & \text{if $\text{class}(a) = \text{class}(b)$;} \\ 0, & \text{otherwise}.\end{cases}$ \;
	$F_{\text{org}}(a, b) = w_{\text{org}}\bigl(\text{commonPrefixLength}(\text{org}(a), \text{org}(b))\bigr)$ \;
	$F_{\text{ref}}(a, b) = w_{\text{ref}}\bigl(\text{number of fields in $a$ that reference $b$}\bigr)$ \;
	$F_{\text{comm}}(a, b) = w_{\text{comm}}\bigl(\text{number of messages from $a$ to $b$}\bigr)$ \;
\end{algorithm}

\begin{table}
	\centering
	\caption{
		Default configuration of force weights for the object graph layout (columns represent assignments).
		References between objects dominate the layout while organizational proximity and communication between objects are weighted lower.
		Users can override these weights for specific program traces.
	}
	\label{tab:visualization_approach/mapping/object_graph/default_configuration}
	\begin{threeparttable}
		\centering
		{\footnotesize
		% !TeX root = ../../../../paper.tex
{\setlength\tabcolsep{3pt}
\begin{tabular}{cccc c cc}
	\toprule

	\multicolumn{4}{c}{\bold{Object-specific forces}}	&
		&
	\multicolumn{2}{c}{\bold{Generic forces}}	\\

	\multicolumn{1}{c}{\bold{$w_{\text{class}}$}}	&
	\multicolumn{1}{c}{\bold{$w_{\text{org}}$}}	&
	\multicolumn{1}{c}{\bold{$w_{\text{ref}}$}}	&
	\multicolumn{1}{c}{\bold{$w_{\text{comm}}$}}	&
		&
	\multicolumn{1}{c}{\bold{$w_{\text{repulse}}$}}	&
	\multicolumn{1}{c}{\bold{$w_{\text{center}}$}}	\\

	\midrule

	$0.001$	&
	$F \mapsto 0.005 \left(\log_{10}(F) + 1\right)$	&
	$0.1$	&
	$0.00001$	&
		&
	$0.2$	&
	$0.00142$	\\

	\bottomrule
\end{tabular}}
}
	\end{threeparttable}
\end{table}

In addition to the attractive forces, we define globally weighted \emph{repulsion} and \emph{centripetation} forces on all blocks to control the entropy of the graph, and we define \emph{radial constraints} to avoid collisions between blocks.

We provide an empirical base configuration for all force weights but allow users to override them for specific program traces.
By default, we give the highest weight to reference forces and the lowest weight to organizational forces with a six-order-of-magnitude difference and scale organizational forces logarithmically~(\cref{tab:visualization_approach/mapping/object_graph/default_configuration}).
This configuration encourages a state-centric layout of the object graph while leaving a margin for the characteristics of particular program traces (e.g., their ratio between intrinsic and extrinsic state~\cite[p. 218ff]{gamma1994design}) towards a more dataflow-driven layout.
In addition, users can drag and drop blocks to customize the layout.
To reduce response time~\cite[chap. 11]{shneiderman2005designing} and maintain an experience of immediacy~\cite{ungar1997debugging}, we render the graph at regular update intervals before the force simulation has converged.

\paragraph{Object selection}
\label{sec:visualization_approach/mapping/object_selection}

Usually, even after restricting the object graph to the receivers from the call tree~(\cref{sec:visualization_approach/data_model}), only a small part of it is relevant for comprehending the high-level behavior of a program while many other objects fulfill lower-level implementation details.
In our visualization, we use a filter system for excluding objects based on their label, class, or organization.
Similar to the layout configuration~(\lcnameref{sec:visualization_approach/mapping/object_graph}), we provide an empirical default configuration that excludes certain base objects such as collections, booleans, and numbers, but allow users to customize these filters.

\paragraph{Object behavior}
\label{sec:visualization_approach/mapping/object_behavior}

The color of each object block indicates its recent activity:
\emph{inactive} blocks are colored in a neutral light gray while \emph{active} blocks whose objects have recently received a message are highlighted in a bright red~(\cref{fig:visualization_approach/mapping/object_behavior}).
After the control flow passes on to other objects, blocks fade linearly back to the base color within one second, thus applying a single-hue continuous sequential color scheme by Brewer et al.\footnote{Cynthia Brewer and Mark Harrower. 2013 -- 2021. ColorBrewer: Color Advice for Cartography. Pennsylvania State University. \textsc{URL}: \url{https://colorbrewer2.org/}}

\begin{figure}
	\includegraphics[width=\linewidth]{sections/03_visualization_approach/mapping/object_behavior}
	\caption{
		Visual mapping of object behavior to color and trail in the object map.
		The intensity of the red color indicates the recency of the last message received by each object.
		The gradient trail curve connects the most recent object activations (control points of the curve are marked with a \protect\texticon{sections/03_visualization_approach/mapping/object_behavior_cross} cross).
	}
	\Description{TODO}
	\label{fig:visualization_approach/mapping/object_behavior}
\end{figure}

In addition to the color coding, a \emph{trail} connects the $k = 15$ most recent object activations to support the delayed observation of short activations and the recognition of the exact activation order.
The trail curve is based on a centripetal Catmull-Rom spline~\cite{catmull1974class} with control points are placed on the top of each relevant block and alternating with intermediate points between blocks.
Block control points are randomized using a normal distribution to distinguish multiple activations of the same object.
Intermediate control points are raised vertically to give the curve a wave-like shape that makes activated objects identifiable.
The direction of the trail is displayed by continuously moving it to the next object during the animation and applying a linear translucency gradient to fade out the tail of the curve.

\paragraph{Timeline}
\label{sec:visualization_approach/mapping/timeline}

The object map integrates a \emph{timeline} overlay at the bottom of the viewport that provides a time-centric navigation aid.
The timeline consists of two widgets stacked on top of each other~(\cref{fig:visualization_approach/mapping/timeline}):
a \emph{player} with a slider and a play/pause button displays the current point in time of the program trace and allows users to control the time and animation state.
Behind the player, a collapsed \emph{flame graph} shows the course of the call stack depth.
Users can resize the timeline to explore the full call tree hierarchy and examine individual frames in the flame graph.

Both the flame graph and the object map are interactively linked; i.e., users can hover over an object in the map to discover all of its activations in the timeline, or vice versa, they can click on a frame to fast-forward or rewind the trail in the map to the corresponding object activation.
Thus, object map and timeline provide two orthogonal means of navigating through the object-oriented program trace at different granularities.

\begin{figure}
	\includegraphics[width=\linewidth]{sections/03_visualization_approach/mapping/timeline}
	\caption{
		Timeline overlay with widgets for controlling the playback of the program trace and a flame graph with a variable level of detail for navigating the call tree.
		The flame graph and the object map are interactively linked, e.g., the user can hover over a frame to highlight the corresponding object in the map.
	}
	\Description{TODO}
	\label{fig:visualization_approach/mapping/timeline}
\end{figure}
