% !TeX root = ../paper.tex
\begin{abstract}
	Program comprehension is a key activity in software development.
	Several visualization approaches such as software maps have been proposed to support programmers in exploring the architecture of software systems, while little attention has been paid to the exploration of program behavior and programmers still rely on traditional code browsing and debugging tools to build a mental model of a system's behavior that connects abstract concepts to implementation artifacts.
	We propose a novel approach to visualizing program behavior through \emph{animated 2.5D object maps} that depict particular objects and their interactions from a program trace.
	We describe our implementation of this approach and evaluate it for different program traces through an experience report and performance measurements.
	Our results indicate that our approach can be beneficial for program comprehension tasks, but that further research is needed to improve scalability and usability.
\end{abstract}
