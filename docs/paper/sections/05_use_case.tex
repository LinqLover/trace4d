% !TeX root = ../paper.tex
\section{Use Case: Exploring Internals of a Regular Expression Engine}
\label{sec:use_case}

In this section, we illustrate how a programmer can use the \tfd{} visualization to explore the way a regular expression engine processes a pattern.
The \code{Regex} package in Squeak provides a Smalltalk-specific flavor of regular expressions.
To process a pattern string, the package first parses it into an abstract syntax tree (AST) and then compiles the AST into a non-deterministic finite automaton (NFA).
In this example, our programmer visualizes the processing of the simple regular expression \hardcode{\textbackslash{}d|\textbackslash{}w+} to gain a closer understanding of the involved subsystems and their interactions.

To create the visualization, the programmer records and exports a trace of the program \code{\textquotesingle{}\textbackslash{}d|\textbackslash{}w+\textquotesingle{} asRegex} in Squeak and loads it into the \tfd{} web app%
\footnote{The interactive visualization of the described program trace is available at \ifanon{\url{https://user.github.io/repo/?trace=traces/regexParse.json} (blinded)}{\url{https://linqlover.github.io/trace4d/?trace=traces/regexParse.json}} and in the Wayback Machine of the Internet Archive.}%
.
As the visualization loads, she can see about 25 objects moving around in the object map and arranging themselves into a semi-structured graph within a few seconds~(\cref{fig:teaser}).
By navigating through the scene, she discovers several meaningful objects and clusters of objects:

\begin{itemize}
	\item the pattern string \code{\textquotesingle{}\textbackslash{}d|\textbackslash{}w+\textquotesingle{}};
	\item an \code{RxParser} object accessing the string through a \code{Read\-Stream};
	\item eight objects referencing each other whose class names start with the prefix \code{Rxs}, identifying them as nodes of the AST;
	\item a \code{RxMatcher} object surrounded by six objects whose class names start with \code{Rxm}, identifying them as states of the matcher's NFA;
	\item several other loosely structured objects, including an \code{Rx\-Match\-Optimizer} object, four \code{Dictionary}s, and a \code{Set}.
\end{itemize}

After she has gained a rough overview of the object graph, she starts the animation of the program trace through the player in the timeline.
By observing the trail of object activations and the position of the cursor in the timeline (default running time: 77 seconds), she notices the following rough segments of the program execution:

\begin{enumerate}
	\item Invoked by the pattern string, the parser dominates the first third of the program, accesses the pattern through the \code{ReadStream}, and talks to the AST nodes, presumably to initialize them.
	\item Next, the matcher gets active and accesses the AST nodes and the NFA states simultaneously, presumably to compile the AST into the NFA.
	\item For the remaining half of the program time, the match optimizer is active, accessing the AST again and talking to the set.
\end{enumerate}

Thus, our programmer was able to gain an initial overview of the different parts of the \code{Regex} package and their collaboration to implement the processing of the input pattern.
Besides, she also noticed that almost 50\si{\percent} of the time were spent in the match optimizer.
Without a closer idea of the role of this object, she suspects this step to be a bottleneck of the processing and wonders whether the optimization might be optional and could be skipped for certain uses of the regular expression.
To dive deeper into the implementation of the \code{Regex} package, she expands the flame graph of the timeline, identifies a few entry point methods of the objects that she found most interesting (e.g., \code{RxParser>>parseStream:} or \code{RxMatchOptimizer>>initialize:ignoreCase:}), and opens them in the Squeak environment to browse their source code.

% TODO: can we do something better with timeline? should we mention configuration?
% TODO: maybe provide link to video of the trace?
