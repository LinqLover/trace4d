% !TeX root = ../paper.tex
\section{Implementation}
\label{sec:implementation}

We demonstrate the feasibility of animated 2.5D object maps by describing the implementation of our prototype \tfd{} that displays program traces from a Squeak/Smalltalk environment (\emph{backend}) in a web application (\emph{frontend}).

\paragraph{Program tracing}
\label{sec:implementation/program_tracing}

Squeak/Smalltalk is an interactive development environment (IDE) that is based on the object-oriented paradigm (everything is an object, including classes, methods, and stack frames) and gives programmers rich control to inspect and manipulate all parts of the system (by instrumenting method objects, recording stack frame objects, etc.)~\cite{ingalls1997back,rowledge2001tour,thiede2023squeak}.
In our backend implementation, we use the \tdb{}\footnote{\ifanon{\url{https://github.com/user/repo} (blinded)}{\url{https://github.com/hpi-swa-lab/squeak-tracedebugger}}}\ifanon{}{~\cite{thiede2023object}}, which is an omniscient debugger for Squeak, to record a program trace of interesting behavior such as compiling a method, matching a string against a regular expression, or handling user events in a graphical user interface (GUI).

We serialize the resulting program trace consisting of a call tree, an object graph, and a class hierarchy and export it to a JSON file.
To retrieve the fields for each object, we use Squeak's built-in inspector tool~\cite[chap. 6, sec. 3]{thiede2023squeak} which collects all instance variables or array elements from each object but also provides higher-level views on the state of known domain objects; for example, the view on a dictionary will omit its internal overallocation array structure and instead display a more comprehensible collection of key-value pairs.
As for the objects referenced as values from fields, we include in the serialization only those objects that receive at least one message in the program trace but only store a flat string representation of all other objects to avoid traversing the entire object graph of the system, most of which is irrelevant to the particular program trace.

\paragraph{Visualization}
\label{sec:implementation/visualization}

We implement the visualization frontend of \tfd{} as a JavaScript web application.
The web app retrieves a serialized program trace and provides a programmatic interface for customizing the visual configuration~(\cref{sec:visualization_approach/mapping/object_graph,sec:visualization_approach/mapping/object_selection}).
To build the 2.5D object map, we generate and display a 3D scene from the program trace using the JavaScript 3D library \name{three.js}\footnote{\url{https://threejs.org/}} and layout the object blocks using the \hardcode{d3-force} module of the visualization framework \name{d3.js}\footnote{\url{https://d3js.org/}}.
To build the timeline, we create a flame graph using the \hardcode{d3-flame-graph} plugin for \name{d3.js}\footnote{\url{https://github.com/spiermar/d3-flame-graph}} and combine it with a custom HTML widget for the player controls%
\footnote{As \hardcode{d3-flame-graph} at the time of writing does not support a notion of starting points but only lengths for frames, we inject auxiliary transparent frames into the flame graph to adjust the horizontal layout of actual frames (see \ifanon{\url{https://gist.github.com/user/id} (blinded)}{\url{https://github.com/spiermar/d3-flame-graph/issues/227}}).}%
.
To animate the visualization, we traverse the call tree at a configurable speed (defaulting to 50 bytecode instructions per second) and update the color highlights and trail for activated objects at each animation tick.
