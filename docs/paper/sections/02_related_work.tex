% !TeX root = ../paper.tex
\section{Related Work}
\label{sec:related_work}

Several approaches for visualizing the architecture and behavior of software systems have been proposed in the past.
In the broad field of program visualization~\cite{myers1986visual,teyseyre2009overview,sorva2013review,merino2018towards}, \emph{algorithm animation} is an early approach that mainly focuses on visualizing procedural algorithms and data structures in educational contexts~\cite{brown1984system}.
During the last decades, more approaches have been proposed that allow to create general-purpose visualizations for the architecture and behavior of arbitrary software systems~\cite{reiss2006visualizing,cheng2008xdiva,chis2014moldable,devkota2022domain}.

\subsection{Software Architecture Visualization}

The term \emph{software maps} describes a family of approaches that use metaphors from cartography to visualize the architecture of software systems.

\paragraph{Treemaps}

\emph{Treemaps} display the static structure of software systems by visualizing their hierarchical organization of packages and classes, folders and files, autc.\footnote{aut cetera: or so on} as a nested set of shapes~\cite{limberger2019advanced,limberger2022visual}.
They offer various visual variables such as the size, color, and position of the shapes to encode additional information about the system's size or evolution.
Shapes are usually rectangles but can also be other polygons as in Voronoi tesselation treemaps~\cite{balzer2005voronoi}.
One popular, contemporary type of treemaps is \emph{2.5D treemaps} which add a third dimension to the visualization by transforming each shape into a right prism (usually a cuboid) of a variable height.
Many approaches use the \emph{software city} metaphor to style the cuboids of a 2.5D treemap as buildings of a city~\cite{dugerdil2008execution,wettel2007visualizing,sasso2015blended,ardigo2021visualizing,mortara2021visualization,hoff2022utilizing,limberger2022visual}.

\paragraph{Thematic software maps}

Unlike treemaps which display the pro\-gram\-mer-specified organization of a software system, \emph{thematic software maps} are a type of \emph{topic maps} that use natural language processing techniques such as source code topic models, latent Dirichlet allocation, and multidimensional scaling to arrange units of the software in a 2D or 3D layout~\cite{atzberger2023visualization}.
Different metaphors have been proposed to embody these graphs in a map, including board games~\cite{atzberger2022visualization} and landscapes such as forests~\cite{atzberger2021softwareforest} and galaxies~\cite{atzberger2021softwaregalaxies}.

\paragraph{Animated software maps}

Next to using static visual variables, some approaches enrich software maps with animations to display dynamic information over time~\cite[sec. 3.4]{lemieux2006visualization}.
Dynamic information can relate to the behavior or evolution of software:
for example, \name{EvoSpaces}~\cite{dugerdil2008execution} highlights classes in a software city when they are activated, while \name{DynaCity}~\cite{dashuber2022trace}, \name{ExplorViz}~\cite{krause2021live}, \name{SynchroVis}~\cite{waller2013synchrovis}, and others~\cite{ciolkowski20173d} also draw connections between modules to visualize dataflow between them;
Langelier et al.~\cite{langelier2008exploring} gradually construct a software city and update the geometries and colors of buildings to represent development activity, and \name{Gource}~\cite{caudwell2010gource} enhances the construction animation of a file tree with moving avatars representing code authors.
Some approaches allow programmers to monitor a system in real time~\cite{fittkau2013live} while others replay a previously recorded trace of software activity~\cite{dugerdil2008execution}.

\subsection{Entity-Centric Behavior Visualization}

To provide visual insights into the behavior of software, a natural choice is to attribute behavior to different entities of the software.
Entities can be organizational units such as modules or classes but also individual object instances of object-oriented programs.

\paragraph{Object graphs}

Several tools allow programmers to explore relevant parts of a program's object graph~\cite{moreno2004visualizing,gestwicki2005methodology}.
Some graphs mimic the look of UML object diagrams and provide details about objects's internal state while others choose more compact representations.
To reduce the visual complexity of graph displays, some tools provide programmers with means for filtering objects based on their organization or relation to program slices~\cite{lange1997object,hamouLhadj2004survey}.

\paragraph{Communication flow}

\emph{Call graphs} and \emph{control-flow graphs} are two popular ways of displaying entities with their mutual dynamic interactions or communications.
Entities can be nodes from an object graph~\cite{tramnitzke2007object}, organizational units such as classes~\cite{reiss2007visual} or modules~\cite{prestin2022hidden}, or subject to user-selected abstraction levels~\cite{lange1997object,dePauw1998execution,walker1998visualizing}.
\name{Avid}~\cite{walker1998visualizing} and \name{PathObjects}~\cite{schweizer2014pathobjects} provide animated object graphs where users can explore the control flow interactively.
Boothe et al. merge the stack frames from a control-flow graph and the nodes from an object graph into a single \emph{memeograph} that can be explored through animation~\cite{boothe2011animation}.

In contrast to traditional call graphs, some works have proposed peripheral, hierarchical layouts of nodes such as \name{Extravis}' \emph{circular bundle views}~\cite{cornelissen2009trace} or the \name{H3} hyperbolic 3D layout~\cite{munzner1997h3}, which provide better scaling for highly connected graphs.
%Another representation of inter-entity communication is to provide full adjacency matrices of the call graph. % TODO: Citation needed! Did not find it!

\paragraph{Dataflow}

Another perspective that can be taken on the object graph is how state is propagated through the system.
The \name{Whyline} approach allows programmers to ask questions about why certain behaviors did or did not happen or where certain values came from and presents the answers in a sliced control-flow graph~\cite{ko2008debugging}.
Lienhard et al. propose an \emph{inter-unit flow view} that displays the amount of data or objects exchanged between different classes or modules in a directed weighted graph~\cite{lienhard2009taking}; this graph can also be embedded into a traditional call graph~\cite{lienhard2009flow}.

\paragraph{State changes}

Lienhard et al. also propose a \emph{side-effects graph}~\cite{lienhard2009flow,fierz2009compass} (also referred to as \emph{test blueprints}~\cite{lienhard2008test}), which shows connections between objects changing each other's state.
Similarly, \emph{object traces} describe a way to slice a call tree~(\cref{sec:related_work/time_centric}) for exploring the state evolution of individual objects~\cite{thiede2023object}.
\emph{Memory cities} support the heap memory analysis of programs by displaying objects and their memory consumption in a 2.5D treemap and animating the allocation and deallocation of objects~\cite{weninger2020memory}.

\subsection{Time-Centric Behavior Visualization}
\label{sec:related_work/time_centric}

Besides the communication or evolution of entities, another perspective that visualizations often take on software behavior is the temporal order of program execution.

\paragraph{Call trees}

A call tree is a hierarchy of stack frames or method invocations that can be obtained from a program trace.
Besides naive graph representations of this data structure, several approaches display call trees using hierarchical layouts such as treemaps, \emph{sunbursts}, or \emph{icicle plots}~\cite{kruskal1983icicle,trumper2012viewfusion,woodburn2019interactive}.
Similarly to icicle plots, \emph{flame graphs} show the historical call stack over time but also assign colors to stack frames to display additional performance data from profiling tools~\cite{gregg2016flame}.

\paragraph{Sequential displays}

UML sequence diagrams are a traditional approach to displaying communication between objects over time.
Several tools adopt~\cite{systä2001shimba} and extend~\cite{hamouLhadj2004survey} this diagram type: for example, \name{ISVis}' \emph{information mural}~\cite{jerding1998information} and \name{Extravis}' \emph{massive sequence view}~\cite{cornelissen2009trace} derive miniaturized versions of a sequence diagram~\cite[sec. 3.4]{lemieux2006visualization}, and \name{Ovation}~\cite{dePauw1998execution} detects execution patterns to reduce sequence diagrams~\cite{hamouLhadj2004survey}.
